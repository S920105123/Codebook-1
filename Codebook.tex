\documentclass[a4paper,10pt,oneside]{article}
\setlength{\columnsep}{15pt}    %兩欄模式的間距
\setlength{\columnseprule}{0pt}

\usepackage[landscape]{geometry}
\usepackage{amsthm}								%定義,例題
\usepackage{amssymb}
\usepackage{fontspec}								%設定字體
\usepackage{color}
\usepackage[x11names]{xcolor}
\usepackage{xeCJK}								%xeCJK
\usepackage{listings}								%顯示code用的
%\usepackage[Glenn]{fncychap}						%排版,頁面模板
\usepackage{fancyhdr}								%設定頁首頁尾
\usepackage{graphicx}								%Graphic
\usepackage{enumerate}
\usepackage{titlesec}
\usepackage{amsmath}
\usepackage{pdfpages}
\usepackage{multicol}
\usepackage{fancyhdr}
%\usepackage[T1]{fontenc}
\usepackage{amsmath, courier, listings, fancyhdr, graphicx}

%\topmargin=0pt
%\headsep=5pt
\textheight=530pt
%\footskip=0pt
\voffset=-20pt
\textwidth=800pt
%\marginparsep=0pt
%\marginparwidth=0pt
%\marginparpush=0pt
%\oddsidemargin=0pt
%\evensidemargin=0pt
\hoffset=-100pt

%\setmainfont{Consolas}				%主要字型
\setCJKmainfont{Microsoft JhengHei}			%中文字型
%\setmainfont{Linux Libertine G}
\setmonofont{Consolas}
%\setmainfont{sourcecodepro}
\XeTeXlinebreaklocale "zh"						%中文自動換行
\XeTeXlinebreakskip = 0pt plus 1pt				%設定段落之間的距離
\setcounter{secnumdepth}{3}						%目錄顯示第三層

\makeatletter
\lst@CCPutMacro\lst@ProcessOther {"2D}{\lst@ttfamily{-{}}{-{}}}
\@empty\z@\@empty
\makeatother
\lstset{											% Code顯示
language=C++,										% the language of the code
basicstyle=\scriptsize\ttfamily, 						% the size of the fonts that are used for the code
numbers=left,										% where to put the line-numbers
numberstyle=\tiny,						% the size of the fonts that are used for the line-numbers
stepnumber=1,										% the step between two line-numbers. If it's 1, each line  will be numbered
numbersep=5pt,										% how far the line-numbers are from the code
backgroundcolor=\color{white},					% choose the background color. You must add \usepackage{color}
showspaces=false,									% show spaces adding particular underscores
showstringspaces=false,							% underline spaces within strings
showtabs=false,									% show tabs within strings adding particular underscores
frame=false,											% adds a frame around the code
tabsize=2,											% sets default tabsize to 2 spaces
captionpos=b,										% sets the caption-position to bottom
breaklines=true,									% sets automatic line breaking
breakatwhitespace=false,							% sets if automatic breaks should only happen at whitespace
escapeinside={\%*}{*)},							% if you want to add a comment within your code
morekeywords={*},									% if you want to add more keywords to the set
keywordstyle=\bfseries\color{Blue1},
commentstyle=\itshape\color{Red4},
stringstyle=\itshape\color{Green4},
}


\newcommand{\includecpp}[2]{
  \subsection{#1}
    \lstinputlisting{#2}
}

\newcommand{\includetex}[2]{
  \subsection{#1}
    \input{#2}
}


\begin{document}
  \begin{multicols}{4}
  \pagestyle{fancy}
  
  \fancyfoot{}
  \fancyhead[L]{National Tsing Hua University - PolarSheep}
  \fancyhead[R]{\thepage}
  
  \renewcommand{\headrulewidth}{0.4pt}
  \renewcommand{\contentsname}{Contents}

   
  \scriptsize
  \section{Data\_Structure}
  \includecpp{hull\_dynamic}{./Data_Structure/hull_dynamic.cpp}
  \includecpp{persistent\_treap}{./Data_Structure/persistent_treap.cpp}
  \includecpp{Treap}{./Data_Structure/Treap.cpp}
  \includecpp{undo\_disjoint\_set}{./Data_Structure/undo_disjoint_set.cpp}
  \includecpp{整體二分}{./Data_Structure/整體二分.cpp}
\section{Flow}
  \includecpp{DFSflow}{./Flow/DFSflow.cpp}
  \includecpp{Dinic}{./Flow/Dinic.cpp}
  \includecpp{min\_cost\_flow}{./Flow/min_cost_flow.cpp}
\section{Geometry}
  \includecpp{circle}{./Geometry/circle.cpp}
  \includecpp{convex\_hull}{./Geometry/convex_hull.cpp}
  \includecpp{geometry}{./Geometry/geometry.cpp}
  \includecpp{KD\_TREE}{./Geometry/KD_TREE.cpp}
  \includecpp{smallest\_circle}{./Geometry/smallest_circle.cpp}
  \includecpp{最近點對}{./Geometry/最近點對.cpp}
\section{Graph}
  \includecpp{3989\_穩定婚姻}{./Graph/3989_穩定婚姻.cpp}
  \includecpp{blossom}{./Graph/blossom.cpp}
  \includecpp{Eulerian\_cycle}{./Graph/Eulerian_cycle.cpp}
  \includecpp{KM}{./Graph/KM.cpp}
  \includecpp{MaximumClique}{./Graph/MaximumClique.cpp}
  \includecpp{MinimumMeanCycle}{./Graph/MinimumMeanCycle.cpp}
  \includecpp{SAT2}{./Graph/SAT2.cpp}
  \includecpp{一般圖最小權完美匹配}{./Graph/一般圖最小權完美匹配.cpp}
  \includecpp{全局最小割}{./Graph/全局最小割.cpp}
  \includecpp{平面圖判定}{./Graph/平面圖判定.cpp}
  \includecpp{最小斯坦納樹DP}{./Graph/最小斯坦納樹DP.cpp}
  \includecpp{最小樹形圖\_朱劉}{./Graph/最小樹形圖_朱劉.cpp}
  \includecpp{穩定婚姻模板}{./Graph/穩定婚姻模板.cpp}
\section{Linear\_Programming}
  \includecpp{simplex}{./Linear_Programming/simplex.cpp}
\section{Number\_Theory}
  \includecpp{basic}{./Number_Theory/basic.cpp}
  \includecpp{bit\_set}{./Number_Theory/bit_set.cpp}
  \includecpp{EXT\_GCD}{./Number_Theory/EXT_GCD.cpp}
  \includecpp{FFT}{./Number_Theory/FFT.cpp}
  \includecpp{find\_real\_root}{./Number_Theory/find_real_root.cpp}
  \includecpp{FWT}{./Number_Theory/FWT.cpp}
  \includecpp{gauss\_elimination}{./Number_Theory/gauss_elimination.cpp}
  \includecpp{LL\_mul}{./Number_Theory/LL_mul.cpp}
  \includecpp{Lucas}{./Number_Theory/Lucas.cpp}
  \includecpp{Matrix}{./Number_Theory/Matrix.cpp}
  \includecpp{Miller\_Rabin}{./Number_Theory/Miller_Rabin.cpp}
  \includecpp{mod\_log}{./Number_Theory/mod_log.cpp}
  \includecpp{NTT}{./Number_Theory/NTT.cpp}
  \includecpp{pollard}{./Number_Theory/pollard.cpp}
\section{String}
  \includecpp{ACA}{./String/ACA.cpp}
  \includecpp{hash}{./String/hash.cpp}
  \includecpp{KMP}{./String/KMP.cpp}
  \includecpp{manacher}{./String/manacher.cpp}
  \includecpp{minimal\_string\_rotation}{./String/minimal_string_rotation.cpp}
  \includecpp{reverseBWT}{./String/reverseBWT.cpp}
  \includecpp{SA}{./String/SA.cpp}
  \includecpp{Z}{./String/Z.cpp}
\section{Tarjan}
  \includecpp{dominator\_tree}{./Tarjan/dominator_tree.cpp}
  \includecpp{橋連通分量}{./Tarjan/橋連通分量.cpp}
  \includecpp{雙連通分量\&割點}{./Tarjan/雙連通分量&割點.cpp}
\section{Tree}
  \includecpp{HLD}{./Tree/HLD.cpp}
  \includecpp{treeDC}{./Tree/treeDC.cpp}
\section{others}
  \includecpp{vimrc}{./others/vimrc.cpp}
\section{zformula}
  \includetex{formula}{./zformula/formula.tex}
  \includetex{java}{./zformula/java.tex}

  \clearpage
  \end{multicols}
  \newpage
  \begin{multicols}{4}
  \enlargethispage*{\baselineskip}
  \begin{center}
    \Huge\textsc{ACM ICPC Team Reference - PolarSheep}
    \vspace{0.35cm}    
  \end{center}
  \tableofcontents
  \end{multicols}
  \clearpage
\end{document}
